
\section{Related Work}

Sample citation to keep Latex happy~\cite{HennessyPatterson:CompArch5thEd:2011}.

Other efforts that exist to solve this problem and why are they less effective than our method
\begin{itemize}
    \item Resist the urge to point out only flaws in other work. Do your best to point out both the strengths and weaknesses to provide as well rounded a view of how your idea relates to other work as possible.
    \item In a social and political sense it is very smart as well as ethically superior to say good things, which are true, about other people’s work. A major motivation for this is that editors and program committee members have to get a set of reviews for your paper. The easiest way for them to decide who should review it is to look at the set of references to related work (e.g., [1,2, 3]) to find people who are likely to be competent to review your paper. The people whose work you talk about are thus likely to be reading what you say about their work while deciding what to say about your work.
    \item  Clear enough? Speak the truth, say what you have to say, but be generous to the efforts of others.
\end{itemize}

Other efforts that exist to solve related problems that are relevant, how are they relevant, and why are they less effective than our solution for this problem.

\begin{itemize}
    \item  Many times no one has solved your exact problem before, but others have solved closely related problems or problems with aspects that are strongly analogous to aspects of your problem.
\end{itemize}

Sometimes, you need to include some equations in your report. Here are some examples that may be useful.

We can write the \emph{quantum state} encoded by a single qubit as
\begin{equation}
    \ket{\psi} \in \{ x : x \in \mathbb{C}^2, \norm{x} = 1 \}
\end{equation}
so that it is a complex-valued vector in 2 dimensions with unit norm. The notation $\ket{\psi}$ is known as \emph{Dirac notation} and is the standard notation for writing quantum states. (We will see later in this section how to visualize a 1-qubit quantum state using a Bloch sphere.) The quantum states
\begin{equation}
\ket{0} = \begin{pmatrix}
1 \\
0
\end{pmatrix}, 
\ket{1}= \begin{pmatrix}
0 \\
1
\end{pmatrix}
\end{equation}
correspond to the classical bit states of 0 and 1. As the astute reader may have observed, these two states form a small subset of possible quantum states. Nevertheless, we still have a tractable method to describe quantum states for the purposes of computation.

This can be done by observing that the set $\{ \ket{0}, \ket{1} \}$ forms a basis for $\mathbb{C}^2$, known as the \emph{computational basis}, so that every quantum state $\ket{\psi}$ can be written as a linear combination of $\ket{0}$ and $\ket{1}$. More formally, we can write
\begin{equation}
\ket{\psi} = \alpha\ket{0} + \beta\ket{1}
\label{eq:qubit-state}
\end{equation}
where $\alpha$ and $\beta$ are \emph{amplitudes}, and
$\lbrace  \alpha,\beta \rbrace \in \mathbb{C}$. When both $\alpha$ and $\beta$ are non-zero, a quantum state is said to be in a \emph{superposition} of $\ket{0}$ and $\ket{1}$~\cite{Nielsen:Quantum:2010}. In this way, we can make sense of the idea that a single qubit is a combination of a $\ket{0}$ and $\ket{1}$ simultaneously. 

